

\documentclass[12pt,a4paper]{report}
\usepackage{graphicx}
\usepackage[utf8]{inputenc}
\usepackage[T1]{fontenc}
\usepackage[a4paper,top=3cm,bottom=2cm,left=3cm,right=3cm,marginparwidth=1.75cm]{geometry}
\usepackage[spanish]{babel}
\selectlanguage{spanish}
\usepackage{fancyhdr}

% quitar el 0.
\renewcommand\thesection{\arabic{section}}

% aqui definimos el encabezado de las paginas pares e impares.
\lhead[x1]{\includegraphics[width=1cm]{./images/epis}}
\chead[y1]{}
\rhead[z1]{Inteligencia de Negocios}
\renewcommand{\headrulewidth}{1pt}

% aqui definimos el pie de pagina de las paginas pares e impares.
\lfoot[a1]{Ing. Patric Cuadros}
\cfoot[c1]{}
\rfoot[e1]{\thepage}
\renewcommand{\footrulewidth}{1pt}

\pagestyle{fancy} 



\begin{document}
\begin{titlepage}
	\centering
	\includegraphics[width=4cm]{./images/upt}\par\vspace{1cm}
	{\scshape\LARGE\huge\bfseries Universidad Privada de Tacna \par}
	{\scshape\LARGE Escuela de Ingenieria de Sistemas \par}
	\vspace{1cm}
	{\scshape\Large Inteligencia de Negocios\par}
	\vspace{0.5cm}
	{\huge\bfseries Diseño de Indicadores ICACIT\par}
	\vspace{1cm}

	{\Large\itshape PRESENTADO POR:\par}
	{\Large\itshape Gary Calle Cortez\par}
	{\Large\itshape Aldo Lopez Mamani\par}
	{\Large\itshape Renzo Moreno\par}
	{\Large\itshape Christian Cespedes Medina\par}
	{\Large\itshape Tommy Morales\par}
	{\Large\itshape Leonardo Acevedo Vasquez\par}
	{\Large\itshape Agregen y completen sus nombres\par}
	\vfill
	Docente\par
	Ing. Patrick Cuadros\textsc{Brown}

	\vfill

% Bottom of the page
	{\large \today\par}

\end{titlepage}

%% Aquí podemos añadir un resumen del trabajo (o del artículo en su caso) 
\begin{abstract}
Esta es una plantilla simple para crear un articulo \LaTeX en español, con algunos comandos que se usarán frecuentemente para hacer tareas de la licenciatura en Física.
\end{abstract}

%% Iniciamos "secciones" que servirán como subtítulos
%% Nota que hay otra manera de añadir acentos


\addcontentsline{toc}{chapter}{Índice general}
\tableofcontents
\newpage

\begin{Large}
\textbf{INDICADORES} 
\end{Large}
\section{Criterio estudiantes}

\subsection{Evaluacion del desempeño y monitoreo del progreso de los estudiantes}
Un indicador visible es el cumplimiento de los PRE-REQUISITOS, esto de analiza con los cursos que presenten pre-requisitos dentro de la base de datos y la aprobacion de estos dentro de los cursos aprobados por cada estudiante
\begin{itemize}
\item Area encargada: Matricula
\item The second item in my list.
\end{itemize}
\subsection{Consejería de Estudiantes}
Un indicador visible es la frecuencia en que los estudiantes acuden a consejeria estudiantil, esto se podra corroborar en la base de datos con su registro al acudir a este servicio
\begin{itemize}
\item Area encargada: Tutoria
\item The second item in my list.
\end{itemize}
\subsection{Traslado de Estudiantes y Transferencia de Cursos}
Un indicador visible son la procedencia de los traslados, todos los requisitos y datos del traslado como su procedencia, pais o departamente es un indicador de donde generalmente solicitan dichos traslados y toda esa dato se almacena en la base de datos
\begin{itemize}
\item Area encargada: Matricula
\item The second item in my list.
\end{itemize}
\subsection{Requisitos de Graduación}
Un indicador visible son los requisitos para la graduacion como el numero de creditos, horas de practicas pre-profesionales, en donde solicitaron las practicas o si varios egresados la hacen regularmente en algun estatal o privado, los datos de el grado del nuevo ingeniero de sistemas se almacenan en distintas tablas relacionadas
\begin{itemize}
\item Area encargada: Egresados
\item The second item in my list.
\end{itemize}

\section{ Criterio Objetivos educacionales del programa}

\section{ Criterio resultados del estudiante}

\section{ Criterio mejora continua}
La mejora continua es un proceso de evolución del estado actual de un proceso, en este caso el de enseñanza. Nuevas teorías y practicas son publicadas constantemente por lo que no debemos quedarnos estancados, debemos buscar adaptarnos a estos cambios ya que estos cambios son para llegar a ser mejores ingenieros.
\begin{itemize}
\item Área Encargada: Docentes, Egresados.
\item Query: SELECT eg.nombre,su.sugerencia FROM egresados eg INNER JOIN sugerencia su ON eg.id = su.egresadoid
Esta consulta se basa en que en la reunión de docentes y egresados, los docentes les preguntan a los egresados que les sirvió mas de lo aprendido en la universidad y que les falto según su experiencia laboral.
\end{itemize}
\section{ Criterio plan de estudios}
El plan de estudio debe especificar áreas temáticas apropiadas para la ingenería e incluir un año de una combinación de \textbf{Matemática de nivel Universitario} y \textbf{Ciencias Básicas}, un año y medio \textbf{Tópicos de Ingeniería} y \textbf{Diseño en ingenería}.

\begin{itemize}
\item Area encargada: Facultad de Ingeniería

\item Posible Consulta: Una consulta o revisión en los cursos que ofrece la universidad donde Matemáticas de Nivel Universitario: Matemáticas por encima del nivel de pre-cálculo
Ciencias Básicas: Consisten en química y física, y otras ciencias naturales incluyendo las ciencias de la vida, de la tierra y espaciales.\\
Diseño en Ingeniería: Proceso creativo, iterativo y de toma de decisiones, en el que las ciencias básicas, las matemáticas y las ciencias de la ingeniería son aplicadas para buscar soluciones viables a un problema que no necesariamente tiene una única respuesta.\\
Select nombre From Cursos where tipo = ING

\item Tabla Involucrada: Cursos
\end{itemize}

\section{ Criterio cuerpo de profesores}


\section{ Oficinas, Salas de Clase y Laboratorios}
\subsection{Evaluacion del desempeño y monitoreo del progreso de los estudiantes}
Un indicador visible es la realizacion de inventariado de los inmuebles, agrando datos como marca, color, fecha de adquisicion, etc.
cada salon de clases esta asociado a distintos equipos como proyectores, ordenadores camaras de seguridad o servidores, ese indicador es muy util
Hay un inidcador mucho mas importante ya que es la realizacion del inventariado y asociacion de las distintas computadoras, servidores con los laboratorios asi mismo con la composicion del hardware de cada implemento
\begin{itemize}
\item Area encargada: Almacen, Asignacion de recursos y seguridad
\item The second item in my list.
\end{itemize}
\subsection{Guía y Orientación}
Un indicador que se renova todos los ciclos es la capacitacion tanto de docentes y alumnos en el uso correcto de las intalacions y aula vitual, esto se da en los primeros ciclos y corroborando la asistencia de dichos alumnos y docentes de manera obligatoria
\begin{itemize}
\item Area encargada: Capacitacion
\item The second item in my list.
\end{itemize}
\subsection{Servicios de Biblioteca}
Es un claro indicador de la frecuencia de usos de los distintos libros, dependiendo de su area (salud, derecho, ingenieria, etc) y la cantidad que son solicitados, esto podra indicar que tipo de libros son mas requeridos constantemente tanto por docentes o estudiantes y que libros no son tan solicitados.
\begin{itemize}
\item Area encargada: Biblioteca, Almacen
\item The second item in my list.
\end{itemize}
\subsection{Recursos Informáticos }
Hay un indicador del tipo de ordenadores y servidores que son utilizados por los alumnos,  asi como los sistemas operativos adecuados y los softwares instalados legalmente
\begin{itemize}
\item Area encargada: Almacen, Asignacion de recursos, seguridad y TI
\item The second item in my list.
\end{itemize}


\section{ Criterio apoyo institucional}

\section{ Criterio investigacion}

\subsection{EJEMPLO DE SUBSECCION}
BLA BLA BLA BLA..........................
Tareas:
\begin{enumerate}
    \item Ingresamos a la página web de Oracle.
    \item Descargamos el programa, aceptamos la licencia y Descargamos el programa .
    \item Ejecutamos el programa..
\end{enumerate}
....................... BLA BLA BLA
\subsection{EJEMPLO DE SUBSECCION}
BLA BLA BLA BLA..........................
....................... BLA BLA BLA
\begin{itemize}
\item The first item in my list. 
\item The second item in my list.
\item The third item in my list.
\end{itemize}
Primero tienes que cargar el archivo de imagen desde su computadora usando el enlace de carga del menú del proyecto. Luego usando el comando 'includegraphics' podrás incluirlo en el documento. Con el entorno de figura y el comando de título podrás agregar un número y un título a la figura. Mira el código de la Figura \ref{fig:tesla} en esta sección para ver un ejemplo.


\end{document}
