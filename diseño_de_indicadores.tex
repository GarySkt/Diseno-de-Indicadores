
\documentclass[12pt,a4paper]{report}
\usepackage{graphicx}

\usepackage[utf8]{inputenc}
\usepackage[T1]{fontenc}
\usepackage[a4paper,top=3cm,bottom=2cm,left=3cm,right=3cm,marginparwidth=1.75cm]{geometry}
\usepackage[spanish]{babel}
\selectlanguage{spanish}

\begin{document}
\begin{titlepage}
	\centering
	\includegraphics[width=4cm]{./images/upt}\par\vspace{1cm}
	{\scshape\LARGE\huge\bfseries Universidad Privada de Tacna \par}
	{\scshape\LARGE Escuela de Ingenieria de Sistemas \par}
	\vspace{1cm}
	{\scshape\Large Inteligencia de Negocios\par}
	\vspace{0.5cm}
	{\huge\bfseries Diseño de Indicadores ICACIT\par}
	\vspace{1cm}

	{\Large\itshape PRESENTADO POR:\par}
	{\Large\itshape Gary Calle Cortez\par}
	{\Large\itshape Aldo Lopez Mamani\par}
	{\Large\itshape Renzo Moreno\par}
	{\Large\itshape Christian Cespedes Medina\par}
	{\Large\itshape Tommi Morales\par}
	{\Large\itshape Agregen y completen sus nombres\par}
	\vfill
	Docente\par
	Ing. Patrick Cuadros\textsc{Brown}

	\vfill

% Bottom of the page
	{\large \today\par}

\end{titlepage}

%% Aquí podemos añadir un resumen del trabajo (o del artículo en su caso) 
\begin{abstract}
Esta es una plantilla simple para crear un articulo \LaTeX en español, con algunos comandos que se usarán frecuentemente para hacer tareas de la licenciatura en Física.
\end{abstract}

%% Iniciamos "secciones" que servirán como subtítulos
%% Nota que hay otra manera de añadir acentos
\section{Criterio ESTUDIANTES}
\subsection{¿Cómo incluir figuras?}
¡Tu introducción va aquí! A continuación, se enumeran algunos ejemplos de comandos y funciones de uso común para ayudarte a comenzar.
\subsection{¿Cómo incluir figuras?}
\section{ Criterio OBJETIVOS EDUCACIONALES DEL PROGRAMA}

\section{ Criterio RESULTADOS DEL ESTUDIANTE}

\section{ Criterio MEJORA CONTINUA}

\section{ Criterio PLAN DE ESTUDIOS  }

\section{ Criterio CUERPO DE PROFESORES}

\section{ Criterio INSTALACIONES}

\section{ Criterio APOYO INSTITUCIONAL}

\section{ Criterio INVESTIGACIÓN}

\subsection{EJEMPLO DE SUBSECCION}
BLA BLA BLA BLA..........................
....................... BLA BLA BLA
\subsection{EJEMPLO DE SUBSECCION}
BLA BLA BLA BLA..........................
....................... BLA BLA BLA
Primero tienes que cargar el archivo de imagen desde su computadora usando el enlace de carga del menú del proyecto. Luego usando el comando 'includegraphics' podrás incluirlo en el documento. Con el entorno de figura y el comando de título podrás agregar un número y un título a la figura. Mira el código de la Figura \ref{fig:tesla} en esta sección para ver un ejemplo.


\end{document}