

\documentclass[12pt,a4paper]{report}
\usepackage{graphicx}
\usepackage[utf8]{inputenc}
\usepackage[T1]{fontenc}
\usepackage[a4paper,top=3cm,bottom=2cm,left=3cm,right=3cm,marginparwidth=1.75cm]{geometry}
\usepackage[spanish]{babel}
\selectlanguage{spanish}
\usepackage{fancyhdr}

% quitar el 0.
\renewcommand\thesection{\arabic{section}}

% aqui definimos el encabezado de las paginas pares e impares.
\lhead[x1]{\includegraphics[width=1cm]{./images/epis}}
\chead[y1]{}
\rhead[z1]{Inteligencia de Negocios}
\renewcommand{\headrulewidth}{1pt}

% aqui definimos el pie de pagina de las paginas pares e impares.
\lfoot[a1]{Ing. Patric Cuadros}
\cfoot[c1]{}
\rfoot[e1]{\thepage}
\renewcommand{\footrulewidth}{1pt}

\pagestyle{fancy} 



\begin{document}
\begin{titlepage}
	\centering
	\includegraphics[width=4cm]{./images/upt}\par\vspace{1cm}
	{\scshape\LARGE\huge\bfseries Universidad Privada de Tacna \par}
	{\scshape\LARGE Escuela de Ingenieria de Sistemas \par}
	\vspace{1cm}
	{\scshape\Large Inteligencia de Negocios\par}
	\vspace{0.5cm}
	{\huge\bfseries Diseño de Indicadores ICACIT\par}
	\vspace{1cm}

	{\Large\itshape PRESENTADO POR:\par}
	{\Large\itshape Gary Calle Cortez\par}
	{\Large\itshape Aldo Lopez Mamani\par}
	{\Large\itshape Renzo Moreno\par}
	{\Large\itshape Christian Cespedes Medina\par}
	{\Large\itshape Tommi Morales\par}
	{\Large\itshape Agregen y completen sus nombres\par}
	\vfill
	Docente\par
	Ing. Patrick Cuadros\textsc{Brown}

	\vfill

% Bottom of the page
	{\large \today\par}

\end{titlepage}

%% Aquí podemos añadir un resumen del trabajo (o del artículo en su caso) 
\begin{abstract}
Esta es una plantilla simple para crear un articulo \LaTeX en español, con algunos comandos que se usarán frecuentemente para hacer tareas de la licenciatura en Física.
\end{abstract}

%% Iniciamos "secciones" que servirán como subtítulos
%% Nota que hay otra manera de añadir acentos


\addcontentsline{toc}{chapter}{Índice general}
\tableofcontents
\newpage

\begin{Large}
\textbf{INDICADORES} \\
\end{Large}
\section{Criterio estudiantes}

\subsection{Evaluacion del desempeño y monitoreo del progreso de los estudiantes}
Un indicador visible es el cumplimiento de los PRE-REQUISITOS, esto de analiza con los cursos que presenten pre-requisitos dentro de la base de datos y la aprobacion de estos dentro de los cursos aprobados por cada estudiante
\subsection{Consejería de Estudiantes}
Un indicador visible es la frecuencia en que los estudiantes acuden a consejeria estudiantil, esto se podra corroborar en la base de datos con su registro al acudir a este servicio
\subsection{Traslado de Estudiantes y Transferencia de Cursos}
Un indicador visible son la procedencia de los traslados, todos los requisitos y datos del traslado como su procedencia, pais o departamente es un indicador de donde generalmente solicitan dichos traslados y toda esa dato se almacena en la base de datos
\subsection{Requisitos de Graduación}
Un indicador visible son los requisitos para la graduacion como el numero de creditos, horas de practicas pre-profesionales, en donde solicitaron las practicas o si varios egresados la hacen regularmente en algun estatal o privado, los datos de el grado del nuevo ingeniero de sistemas se almacenan en distintas tablas relacionadas

\section{ Criterio Objetivos educacionales del programa}

\section{ Criterio resultados del estudiante}

\section{ Criterio mejora continua}

\section{ Criterio plan de estudios}

\section{ Criterio cuerpo de profesores}

\section{ Criterio instalaciones}

\section{ Criterio apoyo institucional}

\section{ Criterio investigacion}

\subsection{EJEMPLO DE SUBSECCION}
BLA BLA BLA BLA..........................
Tareas:
\begin{enumerate}
    \item Ingresamos a la página web de Oracle.
    \item Descargamos el programa, aceptamos la licencia y Descargamos el programa .
    \item Ejecutamos el programa..
\end{enumerate}
....................... BLA BLA BLA
\subsection{EJEMPLO DE SUBSECCION}
BLA BLA BLA BLA..........................
....................... BLA BLA BLA
\begin{itemize}
\item The first item in my list. 
\item The second item in my list.
\item The third item in my list.
\end{itemize}
Primero tienes que cargar el archivo de imagen desde su computadora usando el enlace de carga del menú del proyecto. Luego usando el comando 'includegraphics' podrás incluirlo en el documento. Con el entorno de figura y el comando de título podrás agregar un número y un título a la figura. Mira el código de la Figura \ref{fig:tesla} en esta sección para ver un ejemplo.


\end{document}